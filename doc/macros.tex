% EBNF syntax.

\let\nt\textit % Nonterminal.
\newcommand{\is}{& ${} ::= {}$ &}
\newcommand{\optional}[1]{$[\,\text{#1}\,]$} % Option.
\newcommand{\seplist}[2]{#2#1${}\ldots{}$#1#2}
\newcommand{\sepspacelist}[1]{\seplist{\ }{#1}}
\newcommand{\sepcommalist}[1]{\seplist{,\ }{#1}}
\newcommand{\newprod}{\\\hskip 1cm\barre\hskip2mm}
\newcommand{\phaprod}{\\\hskip 1cm\phantom\barre\hskip2mm}

% Concrete syntax.

\newcommand{\percentpercent}{\kw{\%\%}\xspace}
\newcommand{\deuxpoints}{\kw{:}\xspace}
\newcommand{\barre}{\kw{\textbar}\xspace}
\newcommand{\kangle}[1]{\kw{\textless} #1 \kw{\textgreater}}
\newcommand{\ocamltype}{\kangle{\textit{\ocaml type}}\xspace}
\newcommand{\ocamlparam}{\kangle{\nt{uid} \deuxpoints \textit{\ocaml module type}}\xspace}
\newcommand{\dheader}[1]{\kw{\%\{} #1 \kw{\%\}}}
\newcommand{\dtoken}{\kw{\%token}\xspace}
\newcommand{\dstart}{\kw{\%start}\xspace}
\newcommand{\dtype}{\kw{\%type}\xspace}
\newcommand{\dnonassoc}{\kw{\%nonassoc}\xspace}
\newcommand{\dleft}{\kw{\%left}\xspace}
\newcommand{\dright}{\kw{\%right}\xspace}
\newcommand{\dparameter}{\kw{\%parameter}\xspace}
\newcommand{\dpublic}{\kw{\%public}\xspace}
\newcommand{\dinline}{\kw{\%inline}\xspace}
\newcommand{\dpaction}[1]{\kw{\{} #1 \kw{\}}\xspace}
\newcommand{\daction}{\dpaction{\textit{\ocaml code}}\xspace}
\newcommand{\dprec}{\kw{\%prec}\xspace}
\newcommand{\dequal}{\kw{=}\xspace}
\newcommand{\dquestion}{\kw{?}\xspace}
\newcommand{\dplus}{\kw{+}\xspace}
\newcommand{\dstar}{\kw{*}\xspace}
\newcommand{\dlpar}{\kw{(}\,\xspace}
\newcommand{\drpar}{\,\kw{)}\xspace}
\newcommand{\eos}{\kw{\#}\xspace}
\newcommand{\dnewline}{\kw{\textbackslash n}\xspace}

% Stylistic conventions.

\newcommand{\kw}[1]{\text{\upshape\sf\bfseries #1}}
\newcommand{\inlinesidecomment}[1]{\textit{\textbf{\footnotesize // #1}}}
\newcommand{\sidecomment}[1]{\hskip 2cm\inlinesidecomment{#1}}
\newcommand{\docswitch}[1]{\vspace{1mm plus 1mm}#1.\hskip 3mm}
\newcommand{\error}{\kw{error}\xspace}

% Abbreviations.

\newcommand{\menhir}{Menhir\xspace}
\newcommand{\menhirlib}{\texttt{MenhirLib}\xspace}
\newcommand{\menhirlibconvert}{\href{http://gallium.inria.fr/~fpottier/menhir/convert.mli.html}{\texttt{MenhirLib.Convert}}\xspace}
\newcommand{\menhirinterpreter}{\texttt{MenhirInterpreter}\xspace}
\newcommand{\menhirlibincrementalengine}{\href{http://gallium.inria.fr/~fpottier/menhir/IncrementalEngine.ml.html}{\texttt{MenhirLib.IncrementalEngine}}\xspace}
\newcommand{\menhirlibgeneral}{\href{http://gallium.inria.fr/~fpottier/menhir/General.ml.html}{\texttt{MenhirLib.General}}\xspace}
\newcommand{\cmenhir}{\texttt{menhir}\xspace}
\newcommand{\ml}{\texttt{.ml}\xspace}
\newcommand{\mli}{\texttt{.mli}\xspace}
\newcommand{\mly}{\texttt{.mly}\xspace}
\newcommand{\ocaml}{OCaml\xspace}
\newcommand{\ocamlc}{\texttt{ocamlc}\xspace}
\newcommand{\ocamlopt}{\texttt{ocamlopt}\xspace}
\newcommand{\ocamldep}{\texttt{ocamldep}\xspace}
\newcommand{\ocamlfind}{\texttt{ocamlfind}\xspace}
\newcommand{\make}{\texttt{make}\xspace}
\newcommand{\omake}{\texttt{omake}\xspace}
\newcommand{\ocamlbuild}{\texttt{ocamlbuild}\xspace}
\newcommand{\Makefile}{\texttt{Makefile}\xspace}
\newcommand{\yacc}{\texttt{yacc}\xspace}
\newcommand{\bison}{\texttt{bison}\xspace}
\newcommand{\ocamlyacc}{\texttt{ocamlyacc}\xspace}
\newcommand{\ocamllex}{\texttt{ocamllex}\xspace}
\newcommand{\token}{\texttt{token}\xspace}
\newcommand{\automaton}{\texttt{.automaton}\xspace}
\newcommand{\conflicts}{\texttt{.conflicts}\xspace}
\newcommand{\dott}{\texttt{.dot}\xspace}

% Files in the distribution.

\newcommand{\distrib}[1]{\texttt{#1}}

% Environments.

\newcommand{\question}[1]{\vspace{3mm}$\diamond$ \textbf{#1}}

% Ocamlweb settings.

\newcommand{\basic}[1]{\textit{#1}}
\let\ocwkw\kw
\let\ocwbt\basic
\let\ocwupperid\basic
\let\ocwlowerid\basic
\let\ocwtv\basic
\newcommand{\ocwbar}{\vskip 2mm plus 2mm \hrule \vskip 2mm plus 2mm}
\newcommand{\tcup}{${}\cup{}$}
\newcommand{\tcap}{${}\cap{}$}
\newcommand{\tminus}{${}\setminus{}$}

% Command line options.

\newcommand{\obase}{\texttt{-{}-base}\xspace}
\newcommand{\ocomment}{\texttt{-{}-comment}\xspace}
\newcommand{\odepend}{\texttt{-{}-depend}\xspace}
\newcommand{\orawdepend}{\texttt{-{}-raw-depend}\xspace}
\newcommand{\odump}{\texttt{-{}-dump}\xspace}
\newcommand{\oerrorrecovery}{\texttt{-{}-error-recovery}\xspace}
\newcommand{\oexplain}{\texttt{-{}-explain}\xspace}
\newcommand{\oexternaltokens}{\texttt{-{}-external-tokens}\xspace}
\newcommand{\ofixedexc}{\texttt{-{}-fixed-exception}\xspace}
\newcommand{\ograph}{\texttt{-{}-graph}\xspace}
\newcommand{\oignoreone}{\texttt{-{}-unused-token}\xspace}
\newcommand{\oignoreall}{\texttt{-{}-unused-tokens}\xspace}
\newcommand{\oinfer}{\texttt{-{}-infer}\xspace}
\newcommand{\ointerpret}{\texttt{-{}-interpret}\xspace}
\newcommand{\ointerpretshowcst}{\texttt{-{}-interpret-show-cst}\xspace}
\newcommand{\ologautomaton}{\texttt{-{}-log-automaton}\xspace}
\newcommand{\ologcode}{\texttt{-{}-log-code}\xspace}
\newcommand{\ologgrammar}{\texttt{-{}-log-grammar}\xspace}
\newcommand{\onoinline}{\texttt{-{}-no-inline}\xspace}
\newcommand{\onostdlib}{\texttt{-{}-no-stdlib}\xspace}
\newcommand{\oocamlc}{\texttt{-{}-ocamlc}\xspace}
\newcommand{\oocamldep}{\texttt{-{}-ocamldep}\xspace}
\newcommand{\oonlypreprocess}{\texttt{-{}-only-preprocess}\xspace}
\newcommand{\oonlytokens}{\texttt{-{}-only-tokens}\xspace}
\newcommand{\ostrict}{\texttt{-{}-strict}\xspace}
\newcommand{\osuggestcomp}{\texttt{-{}-suggest-comp-flags}\xspace}
\newcommand{\osuggestlinkb}{\texttt{-{}-suggest-link-flags-byte}\xspace}
\newcommand{\osuggestlinko}{\texttt{-{}-suggest-link-flags-opt}\xspace}
\newcommand{\otable}{\texttt{-{}-table}\xspace}
\newcommand{\otimings}{\texttt{-{}-timings}\xspace}
\newcommand{\otrace}{\texttt{-{}-trace}\xspace}
\newcommand{\ostdlib}{\texttt{-{}-stdlib}\xspace}
\newcommand{\oversion}{\texttt{-{}-version}\xspace}
\newcommand{\ocoq}{\texttt{-{}-coq}\xspace}
\newcommand{\ocoqnocomplete}{\texttt{-{}-coq-no-complete}\xspace}
\newcommand{\ocoqnoactions}{\texttt{-{}-coq-no-actions}\xspace}

% Adding mathstruts to ensure a common baseline.
\newcommand{\mycommonbaseline}{
\let\oldnt\nt
\renewcommand{\nt}[1]{$\mathstrut$\oldnt{##1}}
\let\oldbasic\basic
\renewcommand{\basic}[1]{$\mathstrut$\oldbasic{##1}}
}
